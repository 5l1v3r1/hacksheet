% This work is licensed under the following license:
%
%   Creative Commons Attribution-NonCommercial-ShareAlike 3.0 Unported
%
% Please see the file COPYING for more information.
\documentclass[10pt,landscape]{article}

\usepackage{alltt}
\usepackage{cprotect}
\usepackage{enumitem}
\usepackage[T1]{fontenc}
\usepackage[landscape,margin=13mm,footskip=1pt,includefoot]{geometry}
\usepackage{graphicx}
\usepackage{hyperref}
\usepackage[utf8]{inputenc}
\usepackage{multicol}
\usepackage{setspace}
\usepackage{titlesec}
\usepackage{upquote}
\usepackage{xspace}

\graphicspath{{figs/}}
\pagestyle{empty}
\parindent=0pt
\hypersetup{
    colorlinks=true,        % false: boxed links; true: colored links
    linkcolor=red,          % color of internal links
    citecolor=cyan,         % color of links to bibliography
    filecolor=magenta,      % color of file links
    urlcolor=blue           % color of external links
}
\setitemize{itemsep=1pt,topsep=0pt,parsep=1pt,leftmargin=10pt}
%\titleformat{\subsection}{\sc}{}{}{}
\titlespacing{\subsection}{0pt}{*0}{*1}

\newcommand{\todo}[1]{\textit{\textcolor{red}{TODO: #1}}}
\newcommand{\minisec}[1]{\textsc{#1}\\}
\newcommand{\first}{\emph{(i)}~}
\newcommand{\second}{\emph{(ii)}~}
\newcommand{\third}{\emph{(iii)}~}
\newcommand{\fourth}{\emph{(iv)}~}
\newcommand{\fifth}{\emph{(v)}~}

\newcommand{\os}[1]{\texttt{\footnotesize{#1}}}
\newcommand{\freebsd}{\os{F}}
\newcommand{\bsd}{\os{B}}
\newcommand{\unix}{\os{U}}
\newcommand{\linux}{\os{L}}
\newcommand{\macos}{\os{M}}
\newcommand{\windows}{\os{W}}
\newcommand{\xp}{\os{xp}}
\newcommand{\kk}{\os{2k}}
\newcommand{\kkt}{\os{2k3}}
\newcommand{\kke}{\os{2k8}}
\newcommand{\seven}{\os{7}}

\newenvironment{action}[1]
  {\begin{minipage}[c]{\linewidth}$\star$~#1\begin{itemize}[leftmargin=1cm]}
  {\end{itemize}\end{minipage}\vspace*{3pt}}
\newcommand{\cmd}[2]{\item[#1] {\small\tt\# #2}}
\newcommand{\comment}[1]{\textrm{\small(#1)}}
\newcommand{\bulletpoint}[1]{\item {\small #1}}
\newcommand{\tool}[2]{\item[#1] {\footnotesize\sc{#2}}\xspace}
\newcommand{\button}[1]{\footnotesize\sf#1}
\newcommand{\click}{$\rightarrow$\xspace}

\begin{document}

\begin{multicols*}{3}

{\Huge\scshape
HackSheet\hspace{-2pt}\raisebox{15pt}{\tiny{master}}\hspace{-4pt}
}

%\hfill\includegraphics[width=.4\linewidth]{bro-logo-small}
%\vspace{-38pt}

{\scriptsize
\setstretch{1.5}
\begin{tabular}{l l}
%Version: & \today\\
Author: & \href{http://matthias.vallentin.net}{Matthias Vallentin}\\
Web: & \url{https://github.com/mavam/hacksheet}\\
License: & \href{http://creativecommons.org/licenses/by-nc-sa/3.0/}
                {Attribution-NonCommercial-ShareAlike 3.0 Unported}
\end{tabular}
}

%\hfill
%\href{http://creativecommons.org/licenses/by-nc-sa/3.0/}
%{\includegraphics[width=.2\linewidth]{by-nc-sa}}
%\vspace{-10pt}

\vspace{-10pt}

\section*{Terminology}

Each command contains a list of flags that indicate the OS requirement: Linux
(\linux), BSD (\bsd), FreeBSD (\freebsd), Mac OS (\macos), UNIX (\unix), and
Windows (\windows).

\section*{Reconnaissance}

\subsection*{Scanning}

\begin{action}{Ping sweep of subnet and host range}
\cmd{\unix}{nmap -sP 10.0.0.0/24 192.168.0.128-254}
\end{action}

\begin{action}{Scan specific TCP and UDP ports}
\cmd{\unix}{nmap -pT:21-25,80,U:5000-6000 target}
\end{action}

\begin{action}{TCP SYN scan without connecting}
\cmd{\unix}{nmap -P0 -sS target}
\end{action}

\begin{action}{Detect OS}
\cmd{\unix}{nmap -O target}
\cmd{\unix}{p0f -s trace.pcap}
\end{action}

\begin{action}{Grab application banners}
\cmd{\unix}{nmap -sV target}
\cmd{\unix}{echo QUIT | nc target 1-1024}
\end{action}

\subsection*{Wireless}

\section*{Vulnerability Scanning}

\subsection*{Web}
\begin{action}{Look for web server vulnerabilities}
\cmd{\unix}{nikto -host 10.0.0.1}
\end{action}

\section*{Hardening}

\subsection*{Physical}

\begin{action}{Check devices}
\bulletpoint{Hardware keylogger (e.g., USB dongles)}
\bulletpoint{Rogue WiFi cards}
\end{action}

\subsection*{OS \& Software}

\begin{action}{Check for package repositories}
\cmd{\linux}{vi /etc/apt/sources.list} \comment{Ubuntu}
\cmd{\linux}{vi /etc/yum.repos.d/*} \comment{RHEL/Fedora}
\end{action}

\begin{action}{Run package updates}
\cmd{\linux}{yum upgrade \emph{package}}
\cmd{\linux}{apt-get upgrade \emph{package}}
\end{action}

\begin{action}{Update Kernel}
\cmd{\linux}{yum update kernel} \comment{RHEL/Fedora}
\cmd{\linux}{apt-cache search linux-image; apt-get install
  linux-image-\emph{x.x.x-xx}} \comment{Debian}
\end{action}

\begin{action}{Harden SSHD}
\tool{\unix}{fail2ban}
\cmd{\unix}{vi /etc/ssh/sshd\_config\\
  Protocol 2\\
  AllowUsers root admin webmaster\\
  AllowGroup sshusers\\
  PasswordAuthentication no\\
  HostbasedAuthentication no\\
  RSAAuthentication yes\\
  PubkeyAuthentication yes\\
  PermitEmptyPasswords no\\
  PermitRootLogin no\\
  ServerKeyBits 2048\\
  IgnoreRhosts yes\\
  RhostsAuthentication no\\
  RhostsRSAAuthentication no}
\end{action}

\subsection*{User Management}

\begin{action}{Show account security settings}
\cmd{\unix}{passwd -l \emph{user}}
\cmd{\linux}{chage -l \emph{user}}
\cmd{\windows}{net accounts}
\cmd{\windows}{net accounts /domain}
\end{action}

\begin{action}{Look for users with root privileges}
\cmd{\unix}{awk -F: '\$3 == 0 \{print \$1\}' /etc/passwd}
\end{action}

\begin{action}{Look for users with empty passwords}
\cmd{\unix}{awk -F: '\$2 == "" \{print \$1\}' /etc/shadow}
\end{action}

\begin{action}{Verify group memberships}
\cmd{\unix}{vi /etc/group} \comment{admin, sudo, wheel}
\end{action}

\begin{action}{Check \texttt{sudo} users}
\cmd{\unix}{visudo}
\end{action}

\begin{action}{Check crontab users}
\cmd{\unix}{for u in \$(cut -f1 -d: /etc/passwd); do crontab -u \$u -l; done}
\end{action}

\begin{action}{Check remote authentication}
\cmd{\unix}{vi \textasciitilde/.rhosts}
\cmd{\unix}{vi \textasciitilde/.ssh/*}
\end{action}

\begin{action}{Change passwords}
\cmd{\unix}{pwgen -sy} \comment{generate strong passwords}
\cmd{\unix}{passwd \emph{user}}
\end{action}

\subsection*{File System}

\begin{action}{Secure mount points}
\cmd{\unix}{mount -o nodev,noexec,nosuid /dev.. /tmp}
\end{action}

\begin{action}{List file attributes}
\cmd{\linux}{lsattr /var/log/foo}
\cmd{\bsd}{ls -ol /var/log/foo}
\end{action}

\begin{action}{Make files append-only}
\cmd{\linux}{chattr +a /var/log/foo}
\end{action}

\subsection*{Network}

\begin{action}{Show firewall rules}
\cmd{\linux}{iptables -nL}
\cmd{\linux}{iptables -t nat -nL}
\end{action}

\begin{action}{Check DNS resolver}
\cmd{\unix}{vi /etc/resolv.conf}
\end{action}

\begin{action}{Disable IPv6}
\cmd{\linux}{ipv6.disable=1} \comment{add to kernel line}
\cmd{\linux}{vi /etc/sysctl.conf\\
  net.ipv6.conf.all.disable\_ipv6 = 1\\
  net.ipv6.conf.<interface0>.disable\_ipv6 = 1\\
  net.ipv6.conf.<interfaceN>.disable\_ipv6 = 1\\
  vi /etc/hosts \comment{comment IPv6 hosts}}
\cmd{\linux}{vi /etc/sysconfig/network\\
  NETWORKING\_IPV6=no\\
  IPV6INIT=no\\
  service network restart}
\cmd{\linux}{vi /etc/modprobe.conf\\
  install ipv6 /bin/true \comment{append to file}}
\cmd{\linux}{vi /etc/modprobe.conf  \comment{RHEL/CentOS}\\
  alias net-pf-10 off}
\cmd{\linux}{vi /etc/modprobe.conf \comment{Debian/Ubuntu}\\
  alias net-pf-10 off \\
  alias ipv6 off}
  \cmd{\windows}{reg add
  hklm{\textbackslash}system{\textbackslash}currentcontrolset{\textbackslash}services{\textbackslash}
  tcpip6{\textbackslash}parameters /v DisabledComponents /t REG\_DWORD /d 255}
\end{action}

\begin{action}{Check network configuration}
  \cmd{\linux}{vi /etc/network/interfaces} \comment{Ubuntu}
  \cmd{\linux}{vi /etc/sysconfig/network-scripts/ifcfg-eth*} \comment{RHEL}
\end{action}

\section*{Forensics}

\subsection*{Users}

\begin{action}{Inspect logged in and past users}
\cmd{\unix}{w}
\cmd{\unix}{last | head}
\cmd{\unix}{ps -ef | awk '\$6 != "?"'} (interactive procs)
\end{action}

\subsection*{Processes}

\begin{action}{Inspect startup items}
\cmd{\linux}{initctl show-config}
  (\href{http://upstart.ubuntu.com/cookbook/upstart_cookbook.pdf}{upstart},
  Ubuntu)
\cmd{\freebsd}{less /etc/rc.local} (deprecated)
\cmd{\freebsd}{grep local\_start /etc/defaults/rc.conf}
\tool{\windows}{Autoruns} \button{Everything}
\end{action}

\begin{action}{Find SETUID and SETGID files and types}
\cmd{\unix}{find / \textbackslash( -perm -4000 -o -perm -2000 \textbackslash)
  -exec file \textbackslash\{\textbackslash\} \textbackslash;}
\cmd{\unix}{crontab -e\\
  0 4 * * * find / \textbackslash( -perm -4000 -o -perm -2000 \textbackslash)
  -type f > /var/log/sidlog.new \&\&
  diff /var/log/sidlog.new /var/log/sidlog \&\&
  mv /var/log/sidlog.new /var/log/sidlog}
\end{action}

\begin{action}{Find world/group writeable directories}
\cmd{\unix}{find / \textbackslash( -perm -g+w -o -perm -o+w \textbackslash)
  -type d -exec ls -ald \textbackslash\{\textbackslash\} \textbackslash;}
\end{action}

\begin{action}{Find all unsigned processes}
\tool{\windows}{ProcessExplorer}
\button{Options} \click \button{Verify Image Signatures}
\end{action}

\begin{action}{Display listening TCP/UDP ports}
\cmd{\linux\windows\unix}{netstat -tun}
\cmd{\bsd}{netstat -p tcp -an | egrep 'Proto|LISTEN|udp'}
\cmd{\unix}{lsof -nPi | awk '/LISTEN/'}
\cmd{\freebsd}{sockstat -4 -l}
\end{action}

\begin{action}{Check active connections to find backdoors}
\cmd{\unix}{lsof -nPi | awk '/ESTABLISHED/'}
\end{action}

\subsection*{Cleanup}

\begin{action}{Kill all processes accessing a mount point}
\cmd{\unix}{fuser -k -c /mnt/secret}
\end{action}

\section*{References}
\begin{itemize}
\small
\item \url{http://bit.ly/cmd-line-kung-fu}
\end{itemize}

\end{multicols*}

\end{document}
