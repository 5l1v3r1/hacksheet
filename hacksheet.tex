% This work is licensed under the following license:
%
%   Creative Commons Attribution-NonCommercial-ShareAlike 3.0 Unported
%
% Please see the file COPYING for more information.
\documentclass[10pt,landscape]{article}

\usepackage{alltt}
\usepackage{cprotect}
\usepackage{enumitem}
\usepackage[T1]{fontenc}
\usepackage[landscape,margin=13mm,footskip=1pt,includefoot]{geometry}
\usepackage{graphicx}
\usepackage{hyperref}
\usepackage[utf8]{inputenc}
\usepackage{multicol}
\usepackage{setspace}
\usepackage{titlesec}
\usepackage{upquote}
\usepackage{xspace}
\usepackage{wasysym}

\graphicspath{{figs/}}
\pagestyle{empty}
\parindent=0pt
\hypersetup{
    colorlinks=true,        % false: boxed links; true: colored links
    linkcolor=red,          % color of internal links
    citecolor=cyan,         % color of links to bibliography
    filecolor=magenta,      % color of file links
    urlcolor=blue           % color of external links
}
\setitemize{itemsep=1pt,topsep=0pt,parsep=1pt,leftmargin=10pt}
%\titleformat{\subsection}{\sc}{}{}{}
\titlespacing{\subsection}{0pt}{*0}{*1}

\newcommand{\todo}[1]{\textit{\textcolor{red}{TODO: #1}}}
\newcommand{\minisec}[1]{\textsc{#1}\\}
\newcommand{\first}{\emph{(i)}~}
\newcommand{\second}{\emph{(ii)}~}
\newcommand{\third}{\emph{(iii)}~}
\newcommand{\fourth}{\emph{(iv)}~}
\newcommand{\fifth}{\emph{(v)}~}

\newcommand{\os}[1]{\texttt{\footnotesize{#1}}}
\newcommand{\freebsd}{\os{F}}
\newcommand{\bsd}{\os{B}}
\newcommand{\unix}{\os{U}}
\newcommand{\linux}{\os{L}}
\newcommand{\macos}{\os{M}}
\newcommand{\windows}{\os{W}}
\newcommand{\xp}{\os{xp}}
\newcommand{\kk}{\os{2k}}
\newcommand{\kkt}{\os{2k3}}
\newcommand{\kke}{\os{2k8}}
\newcommand{\seven}{\os{7}}

% TODO: These are not really OSes, we distinguish them. --mavam
\newcommand{\mysql}{\os{mysql}}
\newcommand{\postgres}{\os{psql}}
\newcommand{\win}{\os{sqlcmd}}

\newenvironment{action}[1]
  {\begin{minipage}[c]{\linewidth}$\star$~#1\begin{itemize}[leftmargin=1cm]}
  {\end{itemize}\end{minipage}\vspace*{8pt}}
\newcommand{\cmd}[2]{\item[#1] {\small\tt\# #2}}
\newcommand{\comment}[1]{\textrm{\small(#1)}}
\newcommand{\bulletpoint}[1]{\item {\small #1}}
\newcommand{\tool}[2]{\item[#1] {\footnotesize\sc{#2}}\xspace}
\newcommand{\button}[1]{\footnotesize\sf#1}
\newcommand{\click}{$\rightarrow$\xspace}
\newcommand{\checkbox}{\CheckedBox}

\begin{document}

\begin{multicols*}{3}

{\Huge\scshape
HackSheet\hspace{-2pt}\raisebox{15pt}{\tiny{master}}\hspace{-4pt}
}

%\hfill\includegraphics[width=.4\linewidth]{bro-logo-small}
%\vspace{-38pt}

{\scriptsize
\setstretch{1.5}
\begin{tabular}{l l}
%Version: & \today\\
Author: & \href{https://github.com/berke1337}{BERKE1337}\\
Web: & \url{https://github.com/berke1337/hacksheet}\\
License: & \href{http://creativecommons.org/licenses/by-nc-sa/3.0/}
                {Attribution-NonCommercial-ShareAlike 3.0 Unported}
\end{tabular}
}

%\hfill
%\href{http://creativecommons.org/licenses/by-nc-sa/3.0/}
%{\includegraphics[width=.2\linewidth]{by-nc-sa}}
%\vspace{-10pt}

\vspace{-10pt}

\section*{Terminology}

Each command contains a list of flags that indicate the OS requirement: Linux
(\linux), BSD (\bsd), FreeBSD (\freebsd), Mac OS (\macos), UNIX (\unix), and
Windows (\windows).

\section*{Reconnaissance}

\subsection*{Scanning}

\begin{action}{Ping sweep of subnet and host range}
\cmd{\unix}{nmap -sP 10.0.0.0/24 192.168.0.128-254}
\end{action}

\begin{action}{List all computers in network}
\cmd{\windows}{net view}
\end{action}

\begin{action}{Scan specific TCP and UDP ports}
\cmd{\unix}{nmap -pT:21-25,80,U:5000-6000 target}
\end{action}

\begin{action}{TCP SYN scan without connecting}
\cmd{\unix}{nmap -P0 -sS target}
\end{action}

\begin{action}{Detect OS}
\cmd{\unix}{nmap -O target}
\cmd{\unix}{p0f -s trace.pcap}
\end{action}

\begin{action}{Grab application banners}
\cmd{\unix}{nmap -sV target}
\cmd{\unix}{echo QUIT | nc target 1-1024}
\end{action}

\subsection*{Wireless}

\section*{Vulnerability Scanning}

\subsection*{Web}
\begin{action}{Look for web server vulnerabilities}
\cmd{\unix}{nikto -host 10.0.0.1}
\end{action}

\section*{Hardening}

\subsection*{Physical}

\begin{action}{Check devices}
\bulletpoint{Hardware keylogger (e.g., USB dongles)}
\bulletpoint{Rogue WiFi cards}

\end{action}

\subsection*{OS \& Software}

\begin{action}{Check for suspicious package repositories}
\cmd{\linux}{vi /etc/apt/sources.list} \comment{Ubuntu}
\cmd{\linux}{vi /etc/yum.repos.d/*} \comment{RHEL/Fedora}
\end{action}

\begin{action}{Run package updates}
\cmd{\linux}{yum upgrade \emph{package}}
\cmd{\linux}{apt-get upgrade \emph{package}}
\end{action}

\begin{action}{Update Kernel}
\cmd{\linux}{yum update kernel} \comment{RHEL/Fedora}
\cmd{\linux}{apt-cache search linux-image; apt-get install
  linux-image-\emph{x.x.x-xx}} \comment{Debian}
\end{action}

\begin{action}{Harden SSHD}
\tool{\unix}{fail2ban}
\cmd{\unix}{vi /etc/ssh/sshd\_config\\
  Protocol 2\\
  AllowUsers root admin webmaster\\
  AllowGroup sshusers\\
  PasswordAuthentication no\\
  HostbasedAuthentication no\\
  RSAAuthentication yes\\
  PubkeyAuthentication yes\\
  PermitEmptyPasswords no\\
  PermitRootLogin no\\
  ServerKeyBits 2048\\
  IgnoreRhosts yes\\
  RhostsAuthentication no\\
  RhostsRSAAuthentication no}
\end{action}

\subsection*{User Management}

\begin{action}{Inspect logged in and past users}
\cmd{\unix}{w}
\cmd{\unix}{last | head}
\cmd{\unix}{ps -ef | awk '\$6 != "?"'} (interactive procs)
\tool{\windows}{PsLoggedOn}
\tool{\windows}{Task Manager \click Users Tab}
\cmd{\windows}{wmic computersystem get username}
\cmd{\windows}{wmic /node:\emph{remotecomputer} computersystem \\ get username}
\end{action}

\begin{action}{Show account security settings}
\cmd{\unix}{passwd -l \emph{user}}
\cmd{\linux}{chage -l \emph{user}}
\cmd{\windows}{net accounts}
\cmd{\windows}{net accounts /domain}
\end{action}

\begin{action}{View Users}
\cmd{\windows}{wmic useraccount list brief}
\end{action}

\begin{action}{Look for users with root privileges}
\cmd{\unix}{awk -F: '\$3 == 0 \{print \$1\}' /etc/passwd}
\cmd{\windows}{net localgroup administratos}
\end{action}

\begin{action}{Look for users with empty passwords}
\cmd{\unix}{awk -F: '\$2 == "" \{print \$1\}' /etc/shadow}
\end{action}

\begin{action}{Make passwords expire}
\cmd{\windows}{wmic path Win32\_UserAccount Set PasswordExpires=True}
\cmd{\windows}{wmic path Win32\_UserAccount where name="\emph{username}" Set PasswordExpires=True}
\cmd{\windows}{wmic path /Node:\emph{remotecomputer} Win32\_UserAccount where name="\emph{username}" Set PasswordExpires=True}
\cmd{\linux}{chage -d 0 \emph{username}}
\end{action}

\begin{action}{Set maximum number of login failures}
\cmd{\linux}{faillog -M \emph{maxNumber} -u \emph{username}}
\cmd{\linux}{faillog -r -u \emph{username}}
\cmd{\windows}{net accounts /lockoutthreshold:\emph{maxNumber}}
\cmd{\windows}{net accounts /lockoutduration:\emph{numberOfMinutes}}
\end{action}

\begin{action}{Verify group memberships}
\cmd{\unix}{vi /etc/group} \comment{admin, sudo, wheel}
\end{action}

\newpage

\begin{action}{Check \texttt{sudo} users}
\cmd{\unix}{visudo}
\end{action}

\begin{action}{Check crontab users}
\cmd{\unix}{for u in \$(cut -f1 -d: /etc/passwd); do crontab -u \$u -l; done}
\end{action}

\begin{action}{Check remote authentication}
\cmd{\unix}{vi \textasciitilde/.rhosts}
\cmd{\unix}{vi \textasciitilde/.ssh/*}
\end{action}

\begin{action}{Change passwords}
\cmd{\unix}{pwgen -sy} \comment{generate strong passwords}
\cmd{\unix}{passwd \emph{user}}
\cmd{\windows}{net user \emph{user} *}
\end{action}

\subsection*{File System}

\begin{action}{Secure mount points}
\cmd{\unix}{mount -o nodev,noexec,nosuid /dev.. /tmp}
\end{action}

\begin{action}{List file attributes}
\cmd{\linux}{lsattr /var/log/foo}
\cmd{\bsd}{ls -ol /var/log/foo}
\cmd{\windows}{cacls.exe file.txt}
\end{action}

\begin{action}{File creation date}
\cmd{\windows}{dir /tc /od}
\cmd{\unix}{ls -li /etc | sort -n}
\end{action}

\begin{action}{System file checker }
\cmd{\windows}{sfc /scannow}
\end{action}

\begin{action}{File signature serification}
\cmd{\windows}{sigverif}
\tool{\windows}{sigcheck}
\cmd{\windows}{sigcheck -e -u -s c:{\textbackslash} }
\end{action}

\begin{action}{Make files append-only}
\cmd{\linux}{chattr +a /var/log/foo}
\end{action}

\subsection*{Network}

\begin{action}{Show firewall rules}
\cmd{\linux}{for t in nat mangle filter raw; do iptables -t \$t -nL; done}
\cmd{\windows}{netsh firewall show portopening}
\cmd{\windows}{netsh firewall show allowedprogram}
\cmd{\windows}{netsh firewall show config}
\cmd{\windows}{netsh firewall show state}
\end{action}
\begin{action}{Enable Windows firewall in block mode}
\cmd{\windows}{netsh firewall set opmode mode = enable}
\end{action}
\begin{action}{Add Windows port opening for specific host}
\cmd{\windows}{netsh firewall add portopening protocol = TCP port = 3389 name = RDP mode = ENABLE scope = CUSTOM addresses = 192.168.99.1}
\end{action}
\begin{action}{Remove Windows port opening}
\cmd{\windows}{netsh firewall delete portopening protocol = TCP port = 3389 name = RDP}
\end{action}

\begin{action}{Close ports}
\cmd{\windows}{netsh advfirewall firewall add rule name="BlockAIM" \\
protocol=TCP\\
dir=out remoteport=4099 action=block}
\end{action}

\begin{action} {Shut down SMB vulnerable services}
\tool{\windows}{Seconfig XP}
\checkbox{Disable NetBIOS over TCP/IP (all interfaces)}
\checkbox{Disable SMB over TCP/IP} 
\checkbox{Disable RPC over TCP/IP}
\click \button{Apply}
\click \button{ Yes}
\end{action}




\begin{action}{Check DNS resolver}
\cmd{\unix}{vi /etc/resolv.conf}
\end{action}

\begin{action}{Disable IPv6}
\cmd{\linux}{ipv6.disable=1} \comment{add to kernel line}
\cmd{\linux}{vi /etc/sysctl.conf\\
  net.ipv6.conf.all.disable\_ipv6 = 1\\
  net.ipv6.conf.<interface0>.disable\_ipv6 = 1\\
  net.ipv6.conf.<interfaceN>.disable\_ipv6 = 1\\
  vi /etc/hosts \comment{comment IPv6 hosts}}
\cmd{\linux}{vi /etc/sysconfig/network\\
  NETWORKING\_IPV6=no\\
  IPV6INIT=no\\
  service network restart}
\cmd{\linux}{vi /etc/modprobe.conf\\
  install ipv6 /bin/true \comment{append to file}}
\cmd{\linux}{vi /etc/modprobe.conf  \comment{RHEL/CentOS}\\
  alias net-pf-10 off}
\cmd{\linux}{vi /etc/modprobe.conf \comment{Debian/Ubuntu}\\
  alias net-pf-10 off \\
  alias ipv6 off}
  \cmd{\windows}{reg add
  hklm{\textbackslash}system{\textbackslash}currentcontrolset{\textbackslash}services{\textbackslash}
  tcpip6{\textbackslash}parameters /v DisabledComponents /t REG\_DWORD /d 255}
\end{action}

\begin{action}{Check network configuration}
  \cmd{\linux}{vi /etc/network/interfaces} \comment{Ubuntu}
  \cmd{\linux}{vi /etc/sysconfig/network-scripts/ifcfg-eth*} \comment{RHEL}
\end{action}

\section*{Forensics}

\subsection*{Processes}

\begin{action}{Inspect startup items}
\cmd{\linux}{initctl show-config}
  (\href{http://upstart.ubuntu.com/cookbook/upstart_cookbook.pdf}{upstart},
  Ubuntu)
\cmd{\freebsd}{less /etc/rc.local} (deprecated)
\cmd{\freebsd}{grep local\_start /etc/defaults/rc.conf}
\tool{\windows}{Autoruns} 
\click\button{Options}
\click\button{Filter Options}
\checkbox{Verify code signatures}
\checkbox{Hide Microsoft entries}

\end{action}

\begin{action}{Find SETUID and SETGID files and types}
\cmd{\unix}{find / \textbackslash( -perm -4000 -o -perm -2000 \textbackslash)
  -exec file \textbackslash\{\textbackslash\} \textbackslash;}
\cmd{\unix}{crontab -e\\
  0 4 * * * find / \textbackslash( -perm -4000 -o -perm -2000 \textbackslash)
  -type f > /var/log/sidlog.new \&\&
  diff /var/log/sidlog.new /var/log/sidlog \&\&
  mv /var/log/sidlog.new /var/log/sidlog}
\end{action}

\begin{action}{Find world/group writeable directories}
\cmd{\unix}{find / \textbackslash( -perm -g+w -o -perm -o+w \textbackslash)
  -type d -exec ls -ald \textbackslash\{\textbackslash\} \textbackslash;}
\end{action}

\begin{action}{Find all unsigned processes}
\tool{\windows}{ProcessExplorer}
\button{Options} \click \button{Verify Image Signatures}
\end{action}

\begin{action} {View Process File Location}
\tool{\windows}{ProcessExplorer}
\button{View} \click \button{Select Columns...} \click \button{Image Path}
\end{action}

\begin{action}{Currently Running Tasks/Processes}
\cmd{\windows}{tasklist -svc} 
\cmd{\linux\unix}{ps aux | less}
\cmd{\linux\unix}{top}
\cmd{\linux\unix}{ps -u \emph{user}}
\end{action}

\begin{action}{Kill Tasks/Processes}
\cmd{\windows}{taskkill -pid \emph{pid}}
\cmd{\linux\unix}{kill \emph{pid}}
\end{action}

\subsection*{Network}

\begin{action}{Display listening TCP/UDP ports}
\cmd{\linux\unix}{netstat -plunt}
\cmd{\windows}{netstat -abon | select-string -Context 1, 0 LISTENING}\emph{(PowerShell Only)}
\cmd{\windows}{netstat -aon | findstr LISTENING}\emph{(cmd.exe)}
\tool{\windows}{tcpview}
\cmd{\bsd}{netstat -p tcp -an | egrep 'Proto|LISTEN|udp'}
\cmd{\unix}{lsof -nPi | awk '/LISTEN/'}
\cmd{\freebsd}{sockstat -4 -l}
\end{action}

\begin{action}{Check active connections to find backdoors}
\cmd{\linux}{netstat -punt}
\cmd{\unix}{lsof -nPi | awk '/ESTABLISHED/'}
\end{action} 

\subsection*{Cleanup}

\begin{action}{Kill all processes accessing a mount point}
\cmd{\unix}{fuser -k -c /mnt/secret}
\end{action}

\section*{Miscellaneous}

\subsection*{Date and Time}

\begin{action}{Set date and time}
\cmd{\unix}{date MMddhhmm[[cc]yy]}
\cmd{\windows}{date}
\cmd{\windows}{time}
\end{action}

\subsection*{Network}

\begin{action}{Forward a TCP/UDP port}
\cmd{\unix}{mkfifo f ;\\
nc -l 80 < f | nc 127.0.0.1 6666 > f \&}
\cmd{\linux}{iptables -t nat -A OUTPUT|POSTROUTING \textbackslash\\
  -p tcp -s x.x.x.x --sport 80 -j SNAT \textbackslash\\
  --to-destination 6666}
\cmd{\linux}{iptables -t nat -A INPUT|PREROUTING \textbackslash\\
  -p tcp -d x.x.x.x --dport 80 -j DNAT \textbackslash\\
  --to-destination :6666}
% Can we do this with -j REDIRECT as well?
%iptables -A PREROUTING -i eth1 -p tcp -m tcp --dport 80 -j REDIRECT --to-port 8080
\end{action}

\subsection*{Databases}

\begin{action}{Export/Restore}
\cmd{\mysql}{mysqldump -u \emph{username} -p \emph{database\_name} > dump.sql}
\cmd{\mysql}{mysql -u \emph{username} -p \emph{database\_name} < dump.sql}
\cmd{\postgres}{pg\_dump \emph{database\_name} > dump.sql}
\cmd{\postgres}{psql -d \emph{database\_name} -f dump.sql}
\end{action}
\begin{action}{Change user password}
\cmd{\mysql}{SET PASSWORD FOR 'root' = PASSWORD('new-pass'); FLUSH PRIVILEGES;}
\cmd{\postgres}{ALTER USER root WITH PASSWORD 'new-pass';}
\cmd{\win}{ALTER LOGIN user WITH PASSWORD = 'pass'; GO;}
\end{action}
\begin{action}{Add/Delete user}
\cmd{\mysql}{CREATE USER 'user'@'localhost' IDENTIFIED BY 'pass';}
\cmd{\mysql}{DROP USER user;}
\cmd{\postgres}{CREATE USER user-name WITH PASSWORD 'pass' VALID UNTIL 'Jan 1 2014';}
\cmd{\postgres}{DROP USER user-name;}
\end{action}
\begin{action}{Permissions}
\cmd{\mysql}{GRANT ALL ON db1.* TO 'foo'@'localhost'; FLUSH PRIVILEGES;}
\cmd{\mysql}{GRANT SELECT ON db2.invoice TO 'bar'@'localhost'; FLUSH PRIVILEGES;}
\cmd{\mysql}{REVOKE ALL ON *.* TO 'bar'@'localhost'; FLUSH PRIVILEGES;}
\cmd{\postgres}{GRANT ALL PRIVILEGES ON *.* TO user;}
\cmd{\postgres}{REVOKE ALL PRIVILEGES ON *.* FROM user;}
\cmd{\win}{GRANT ALL PRIVILEGES ON *.* TO windows-db-user [WITH GRANT OPTION]; GO;}
\cmd{\win}{GRANT SELECT ON *.* TO user; GO;}
\cmd{\win}{USE db-name; REVOKE ALL PRIVILGES FROM user; GO;}
\cmd{\win}{USE db-name; REVOKE [GRANT OPTION FOR] ALTER FROM user; GO;}
\end{action}


\subsection*{Windows Tasks}

\begin{action}{Open Network Connections}
\cmd{\windows}{ncpa.cpl}
\end{action}
\begin{action}{Open Basic Firewall}
\cmd{\windows}{firewall.cpl}
\end{action}
\begin{action}{Open Advanced Firewall}
\cmd{\windows}{wf.msc}
\end{action}
\begin{action}{Open Internet Options}
\cmd{\windows}{inetcpl.cpl}
\end{action}
\begin{action}{Open Programs \& Features}
\cmd{\windows}{appwiz.cpl}
\end{action}
\begin{action}{Open Local User Manager}
\cmd{\windows}{lusrmgr.msc}
\end{action}

\begin{action}{Download file from Internet}
\cmd{\windows}{ 
Powershell \\
\$source = "http:www.download.com/file.txt" \\
\$destination = "c:\textbackslash temp\textbackslash file.txt" \\
\$wc = New-Object System.Net.WebClient \\
\$wc.DownloadFile(\$source, \$destination)}
\end{action}
  
\begin{action}{List device drivers and their properties}
\cmd{\windows}{driverquery (-v)}
\end{action}

\subsection*{OpenSSL Certificate Manipulation}

\begin{action}{Create a self-signed certificate}
    \cmd{\unix}{openssl req -x509 -nodes -days 365 -newkey rsa:2048 -keyout
    priv.key -out cert.crt}
\end{action}
\begin{action}{Create a private key and CSR}
    \cmd{\unix}{openssl req -out CSR.csr -new -newkey rsa:2048 -nodes -keyout
    priv.key}
\end{action}
\begin{action}{Create CSR for an existing private key}
    \cmd{\unix}{openssl req -out CSR.csr -key priv.key -new}
\end{action}
\begin{action}{Create a CSR for an existing certificate}
    \cmd{\unix}{openssl x509 -x509toreq -in cert.crt -out CSR.csr
    -signkey priv.key}
\end{action}
\begin{action}{Remove passphrase from a private key}
    \cmd{\unix}{openssl rsa -in priv.pem -out new\_priv.pem}
\end{action}
\begin{action}{Inspect a CSR}
    \cmd{\unix}{openssl req -text -noout -verify -in CSR.csr}
\end{action}
\begin{action}{Inspect a private key}
    \cmd{\unix}{openssl rsa -in priv.key -check}
\end{action}
\begin{action}{Inspect a certificate}
    \cmd{\unix}{openssl x509 -in cert.crt -text -noout}
\end{action}
\begin{action}{Inspect a PKCS\#12 file (.pfx or .p12)}
    \cmd{\unix}{openssl pkcs12 -info -in keyStore.p12}
\end{action}

\section*{References}

\begin{itemize}
\small
\item \url{http://bit.ly/cmd-line-kung-fu}
\item \url{http://bit.ly/useful-windows-one-liners}
\item \url{http://bit.ly/vmware-esxi-reference}
\item \url{http://bit.ly/ssl-commands}
\end{itemize}

\section*{Tool Downloads}

\begin{itemize}
\small
\item Sys Internals: \url{http://bit.ly/sys-internals}
\item Seconfig XP: \url{http://seconfig.sytes.net/}
\end{itemize}

\end{multicols*}

\end{document}
